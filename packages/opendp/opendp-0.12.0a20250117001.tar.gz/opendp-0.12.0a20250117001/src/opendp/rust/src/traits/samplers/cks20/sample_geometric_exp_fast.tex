\documentclass{article}
% common styling and macros shared by all proof files

\usepackage[top=1in, right=1in, left=1in, bottom=1.5in]{geometry}

\usepackage{amsmath,amsthm,amsfonts,amssymb,amscd}
\usepackage{listings}
\usepackage{hyperref}
\usepackage{xcolor}
\usepackage{xr}

\usepackage{enumerate} 
\usepackage{physics}
\usepackage{fancyhdr}
\usepackage{hyperref}
\usepackage{graphicx}
\usepackage{tcolorbox}
\usepackage{catchfile}
\usepackage{pdftexcmds}
\usepackage[T1]{fontenc}

% hyperref
\hypersetup{
  colorlinks=true,
  linkcolor=blue,
  linkbordercolor={0 0 1}
}

% \contrib macro to indicate inclusion in "contrib".
\usepackage{tcolorbox}
\newtcolorbox{warn_box}{colback=red!5!white,colframe=red!75!black}
\newcommand{\contrib}{{\begin{warn_box}This proof resides in \textbf{``contrib''} because it has not completed the vetting process.\end{warn_box}}} 
\newcommand{\floatingPoint}{{\begin{warn_box}This implementation is susceptible to floating-point vulnerabilities.\end{warn_box}}} 

% asOfCommit macro to version a code dependency. Arguments:
%    #1: relative path to file you are dependent on
%    #2: commit hash it was last edited. If outdated, this should be the second hash in the footnoote. Otherwise,
%            git log -n 1 --pretty=format:%h -- path/to/file.rs
\makeatletter
\ifnum\pdf@shellescape=1
   % "private" command that builds a link to a blob
  \newcommand{\linkOpendpBlob}[3]{%
    \href{https://github.com/opendp/opendp/blob/#1/#2#3}{\path{#3} at commit #1}}

  % latex macro expansion has a separate phase for \input evaluation
  %     immediately evaluate a command to write a temp file to ./out containing the current directory
  \immediate\write18{[ ! -f out/cwd.txt ] && (mkdir -p out && git rev-parse --show-prefix | sed "s|_|\@backslashchar\@backslashchar\@backslashchar_|g" > out/cwd.txt)}
  %     ...and then retrieve the current working directory by loading the temp file
  \CatchFileDef\GitWorkingDir{out/cwd.txt}{\endlinechar=-1}

  % command for building the (up to date) or (outdated) status
  \newcommand{\fileStatus}[2]{%
  \setbox0=\hbox{\input|"git --no-pager log -n1 --pretty='\@percentchar H' #1 | grep -E '^#2.*'"\unskip}\ifdim\wd0=0pt
        (outdated\footnote{See new changes with \texttt{git diff #2..\input|"git --no-pager log -n1 --pretty='\@percentchar h' #1" \GitWorkingDir\path{#1}}})\else
        (up to date)\fi
  }

  \newcommand{\asOfCommit}[2]{%
      % permalink the target
      \linkOpendpBlob{#2}{\GitWorkingDir}{#1}
      % conditionally add (outdated) or (up to date) depending on matching commit hash
      \fileStatus{#1}{#2}%
  }
\else
  % simplified command if shell-escape not enabled
  \newcommand{\asOfCommit}[2]{#1 at commit #2 (unknown status\footnote{Shell-escape is not enabled. Enable \texttt{--shell-escape} to check if this proof is up-to-date with the code.})}
\fi
\makeatother

% \vettingPR macro to link a PR. Arguments:
%    #1: PR number
\newcommand{\vettingPR}[1]{\href{https://github.com/opendp/opendp/pull/#1}{Pull Request \##1}}

% for links to rustdoc items in OpenDP. Arguments:
%    #1: path to item, and designation as trait, struct, fn, etc.
%    #2: item name
\makeatletter
\ifnum\pdf@shellescape=1
  % latex macro expansion has a separate phase for \input evaluation
  %     immediately evaluate a command to write a temp file to ./out containing the base path
  \immediate\write18{[ ! -f out/rustdoc.txt ] && mkdir -p out && ([ -z `kpsewhich --var-value OPENDP_RUSTDOC_PORT` ] && echo "https://docs.rs/opendp/`head -n 1 \@backslashchar`git rev-parse --show-toplevel\@backslashchar`/VERSION | sed 's|.*-dev.*|latest|g'`" || echo "http://localhost:`kpsewhich --var-value OPENDP_RUSTDOC_PORT`") > out/rustdoc.txt}
  %     ...and then retrieve the base path by loading the temp file
  \CatchFileDef\OpenDPRustdocBase{out/rustdoc.txt}{\endlinechar=-1}
\else
  % if shell commands are not enabled, just claim latest
  \newcommand{\OpenDPRustdocBase}{https://docs.rs/opendp/latest}
\fi
\makeatother
\newcommand{\rustdoc}[2]{\href{\OpenDPRustdocBase/opendp/#1.#2.html}{\texttt{#2}}}

% for links to external dependencies. Arguments:
%    #1: crate name
%    #2: path to item, and designation as trait, struct, fn, etc.
%    #3: item name
\newcommand{\docsrs}[3]{\href{https://docs.rs/#1/latest/#1/#2.#3.html}{\texttt{#3}}}

% minted (pseudocode)
\definecolor{codegreen}{rgb}{0,0.6,0}
\definecolor{codegray}{rgb}{0.5,0.5,0.5}
\definecolor{codepurple}{rgb}{0.58,0,0.82}
\definecolor{backcolour}{rgb}{0.95,0.95,0.92}

\lstdefinestyle{mystyle}{
    backgroundcolor=\color{backcolour},   
    commentstyle=\color{codegreen},
    keywordstyle=\color{magenta},
    numberstyle=\tiny\color{codegray},
    stringstyle=\color{codepurple},
    basicstyle=\ttfamily\footnotesize,
    breakatwhitespace=false,         
    breaklines=true,                 
    captionpos=b,                    
    keepspaces=true,                 
    numbers=left,                    
    numbersep=5pt,                  
    showspaces=false,                
    showstringspaces=false,
    showtabs=false,                  
    tabsize=2
}

\lstset{style=mystyle}

% common commands
\theoremstyle{definition}
\newtheorem{theorem}{Theorem}[section]
\newtheorem{lemma}[theorem]{Lemma}
\newtheorem{definition}[theorem]{Definition}
\newtheorem{warning}{Warning}
\newtheorem{corollary}{Corollary}
\newtheorem{proposition}{Proposition}
\newtheorem{remark}{Remark}
\newtheorem{observation}{Observation}
\newtheorem{note}{Note}

\newcommand{\vicki}[1]{{ {\color{olive}{(vicki)~#1}}}}
\newcommand{\hanwen}[1]{{ {\color{purple}{(hanwen)~#1}}}}
\newcommand{\zach}[1]{{ {\color{red}{(zach)~#1}}}}

\newcommand{\MultiSet}{\mathrm{MultiSet}}
\newcommand{\len}{\mathrm{len}}
\newcommand{\din}{\texttt{d\_in}}
\newcommand{\dout}{\texttt{d\_out}}
\newcommand{\T}{\texttt{T} }
\newcommand{\F}{\texttt{F} }
\newcommand{\Map}{\texttt{Map}}
\newcommand{\X}{\mathcal{X}}
\newcommand{\Y}{\mathcal{Y}}
\newcommand{\True}{\texttt{True}}
\newcommand{\False}{\texttt{False}}
\newcommand{\clamp}{\texttt{clamp}}
\newcommand{\function}{\texttt{function}}
\newcommand{\float}{\texttt{float }}
\newcommand{\questionc}[1]{\textcolor{red}{\textbf{Question:} #1}}


\newcommand{\validTransformation}[2]{%
  \begin{theorem}
  For every setting of the input parameters #1 to #2 such that the given preconditions
  hold, #2 raises an exception (at compile time or run time) or returns a valid transformation. A valid transformation has the following properties:
  \begin{enumerate}
      \item \textup{(Appropriate output domain).} 
      For every element $x$ in \texttt{input\_domain}, $\function(x)$ is in \texttt{output\_domain} or raises a data-independent runtime exception.
      
      \item \textup{(Stability guarantee).} 
      For every pair of elements $x, x'$ in \texttt{input\_domain} and for every pair $(\din, \dout)$, 
      where \din\ has the associated type for \texttt{input\_metric} and \dout\ has the associated type for \\ \texttt{output\_metric}, 
      if $x, x'$ are \din-close under \texttt{input\_metric}, $\texttt{stability\_map}(\din)$ does not raise an exception,
      and $\texttt{stability\_map}(\din) \leq \dout$, 
      then $\function(x), \function(x')$ are $\dout$-close under \texttt{output\_metric}.
  \end{enumerate}
  \end{theorem}
}


\newcommand{\validMeasurement}[2]{%
  \begin{theorem}
  For every setting of the input parameters #1 to #2 such that the given preconditions
  hold, #2 raises an exception (at compile time or run time) or returns a valid measurement. A valid measurement has the following property:
  \begin{enumerate}
      \item \textup{(Privacy guarantee).}
      For every pair of elements $x, x'$ in \texttt{input\_domain} and for every pair $(\din, \dout)$,
      where \din\ has the associated type for \texttt{input\_metric} and \dout\ has the associated type for \\ \texttt{output\_measure},
      if $x, x'$ are \din-close under \texttt{input\_metric}, $\texttt{privacy\_map}(\din)$ does not raise an exception,
      and $\texttt{privacy\_map}(\din) \leq \dout$,
      then $\function(x), \function(x')$ are $\dout$-close under \texttt{output\_measure}.
  \end{enumerate}
  \end{theorem}
}


\title{\texttt{fn sample\_geometric\_exp\_fast}}
\author{Michael Shoemate}

\begin{document}
\maketitle

\contrib
Proves soundness of \texttt{fn sample\_geometric\_exp\_fast} in \asOfCommit{mod.rs}{0be3ab3e6}.
This proof is an adaptation of \href{https://arxiv.org/pdf/2004.00010.pdf#subsection.5.2}{subsection 5.2} of \cite{CKS20}.

\subsection*{Vetting history}
\begin{itemize}
    \item \vettingPR{519}
\end{itemize}

\section{Hoare Triple}
\subsection*{Precondition}
$\texttt{x} \in \mathbb{Q} \land \texttt{x} > 0$

\subsection*{Pseudocode}        
\lstinputlisting[language=Python,firstline=2,escapechar=|]{./pseudocode/sample_geometric_exp_fast.py}

\subsection*{Postcondition}
\label{postcondition}
For any setting of the input parameter \texttt{x} such that the given preconditions hold, \\
\texttt{sample\_geometric\_exp\_fast} either returns \texttt{Err(e)} due to a lack of system entropy,
or \texttt{Ok(out)}, where \texttt{out} is distributed as $Geometric(1 - exp(-x))$.

\section{Proof}
Assume the preconditions are met.

\begin{lemma}\label{err-e}
    \texttt{sample\_geometric\_exp\_fast} only returns \texttt{Err(e)} when there is a lack of system entropy.
\end{lemma}

\begin{proof}
    \texttt{x} is of type \docsrs{rug}{struct}{Rational}, there exists some non-negative integer $s$ and positive integer $t$ such that $x = s/t$.
    This is why \texttt{Rational.into\_numer\_denom} is infallible.
    Since $t$ is a positive integer, the preconditions on \rustdoc{traits/samplers/uniform/trait}{SampleUniformIntBelow} are met, 
    \texttt{sample\_uniform\_int\_below} can only return an error due to lack of system entropy, and $u$ is a non-negative integer.
    Similarly, the preconditions on \rustdoc{traits/samplers/cks20/fn}{sample\_bernoulli\_exp} and \rustdoc{traits/samplers/cks20/fn}{sample\_geometric\_exp\_slow} are met,
    and their definitions guarantee an error is only returned due to lack of system entropy.
    The only source of errors is from the invocation of these functions,
    therefore \texttt{sample\_geometric\_exp\_fast} only returns \texttt{Err(e)} when there is a lack of system entropy.
\end{proof}

We now establish some lemmas that will be useful in proving the distribution of \texttt{out}.

\begin{itemize}
    \item Let $u$ be a realization of a random variable $U \sim Uniform(0, t)$, supported on $[0, t)$.
    \item Let $d$ be a realization of a random variable $D \sim Bernoulli(exp(-u/t))$
    \item Let $v$ be a realization of a random variable $V \sim Geometric(1 - exp(-1))$
\end{itemize}

\begin{lemma}\label{geom_1_t}\cite{CKS20}
    Conditioned on $d = \top$, if $z = u + t \cdot v$, 
    then $z$ is a realization of a random variable $Z \sim Geometric(1 - exp(-1/t))$. 
    Equivalently, $P[Z=z | D=\top] = (1 - e^{-1/t}) e^{-z/t}$.
\end{lemma}

\begin{proof}
    For any z, define $u_z := z \ mod \ t$ and $v_z := \lfloor z/t \rfloor$, so that $z = u_z + t \ v_z$. 

    \begin{align*}
        P[Z=z | D=\top] &= P[U = u_z, V = v_z | D = \top] && \text{since } z = u_z + t \cdot v_z \\
        &= P[U = u_z | D = \top] P[V = v_z] && \text{as $U$ and $V$ are independent}\\
        &= \frac{P[U = u_z]}{P[D=\top]} P[D=\top | U = u_z] \cdot (1 - e^{-1}) e^{-v_z} && \text{by Bayes' theorem}\\
        &= \frac{1/t}{1/t \sum_{k=0}^{t-1}e^{-k/t}} e^{-u_z/t} \cdot (1 - e^{-1}) e^{-v_z} && \text{since } P[D = \top] = \frac{1}{t} \sum_{k=0}^{t-1}e^{-k/t} \\
        &= \frac{(1 - e^{-1})}{\sum_{k=0}^{t-1}e^{-k/t}}  e^{-(u_z/t + v_z)} \\
        &= (1 - e^{-1/t}) e^{-(u_z/t + v_z)} \\
        &= (1 - e^{-1/t}) e^{-z/t} && \text{since } z = u_z + t \cdot v_z
    \end{align*}
\end{proof}



\begin{lemma}\label{divide_geometric}\cite{CKS20}
    Fix $p \in (0, 1]$. Let G be a $Geometric(1 - p)$ random variable, and $n \geq 1$ be an integer. 
    Then $\lfloor G / n \rfloor$ is a $Geometric(1 - q)$ random variable with $q = p^n$.
\end{lemma}

\begin{proof}
    \begin{align*}
        P[\lfloor G/n \rfloor = k] &= P[nk < G < (k + 1)n] &&\text{any $G$ in the interval maps to $k$} \\
        &= \sum_{l=kn}^{(k+1)n - 1} (1 - p)p^l \\
        &= (1 - p^n)p^{nk} \\
        &= (1 - q)q^k
    \end{align*}
\end{proof}

\begin{theorem} \label{geom_s_t} \cite{CKS20}
    Given any $s,t \in \mathbb{Z}_+$ and $Z \sim Geometric(1-exp(-1/t))$, 
    define $Y = \lfloor Z / s \rfloor$.
    Then $Y \sim Geometric(1 - exp(-s/t))$.
\end{theorem}

\begin{proof}
    \begin{align*}
        P[Y = y | D = \top] &= P[\lfloor Z/s \rfloor = y | D = \top] \\
        &= (1 - p^s) p^{sk} && \text{by } \ref{divide_geometric} \\
        &= (1 - (e^{-1/t})^s) (e^{-1/t})^{sk} \\
        &= (1 - e^{-s/t}) (e^{-s/t})^k
    \end{align*}
\end{proof}


\begin{lemma}\label{ok-out}
    If the outcome of \texttt{sample\_geometric\_exp\_fast} is \texttt{Ok(out)}, 
    then \texttt{out} is distributed as $Geometric(1 - exp(-x))$.
\end{lemma}

\begin{proof}
    As shown in \ref{err-e}, the preconditions for \rustdoc{traits/samplers/uniform/trait}{SampleUniformIntBelow} on line \ref{line:U},
    \rustdoc{traits/samplers/cks20/fn}{sample\_bernoulli\_exp} on line \ref{line:D},
    and \rustdoc{traits/samplers/cks20/fn}{sample\_bernoulli\_exp\_slow} on line \ref{line:V} are met.
    Therefore, \texttt{u}, \texttt{d} and \texttt{v} follow the distributions necessary to apply \ref{geom_1_t}.
    By \ref{geom_1_t}, $\mathtt{z}$ is a realization of $Z \sim Geometric(1 - exp(-1/t)$. 
    Since \texttt{z} is a realization of $Z \sim Geometric(1 - exp(-1/t))$, 
    then by \ref{geom_s_t}, \texttt{out} is distributed as $Geometric(1 - exp(-x))$.    
\end{proof}

\begin{proof}
    \ref{postcondition} holds by \ref{err-e} and \ref{ok-out}.
\end{proof}

\bibliographystyle{alpha}
\bibliography{mod}
\end{document}